\documentclass{article}%
\usepackage[T2A]{fontenc}%
\usepackage[utf8]{inputenc}%
\usepackage{lmodern}%
\usepackage{textcomp}%
\usepackage{lastpage}%
\usepackage[russian]{babel}%
\usepackage{titling}%
\usepackage{nopageno}%
\usepackage{amsfonts}%
\usepackage{amssymb}%
\usepackage{amsmath}%
\usepackage[left=2cm,right=2cm, top=2cm,bottom=2cm]{geometry}%
\usepackage{circuitikz}%

\begin{document}%

\begin{titlepage}
  \begin{center}
    \large
    Министерство образования Республики Беларусь\\
    \vspace{0.5cm}
    БЕЛОРУССКИЙ ГОСУДАРСТВЕННЫЙ УНИВЕРСИТЕТ
    \vspace{0.5cm}
     
    Факультет прикладной математики и информатики
\bigskip
\vfill
\vfill
\vfill
\vfill
\centerline{\Large \bf Лабораторная работа 1}
    \vspace{0.5cm}
\centerline{"Численные методы решения задачи Коши для ОДУ"}
\end{center}

\vspace*{\fill}
\vfill
\vfill
\vfill
\hfill
\begin{minipage}{0.25\textwidth}
{   Подготовил:\\ студент 3 курса 3 группы\\ Тев Никита Михайлович\\}
\end{minipage}

\mbox{}
\vfill
\hfill
\begin{minipage}{0.25\textwidth}
  {\large{\bf Преподаватель: } 
{\it\\ Горбачёва Юлия \\ Николаевна}}
\end{minipage}

\vspace*{\fill}
\vfill
\vfill
\vfill
\vfill
\vfill
\vfill
\vfill
\vfill
\vfill
\vspace*{\fill}
\begin{center}
Минск, 2019 г.
\end{center}
\end{titlepage}

\begin{enumerate}%
\item%
\underline{Постановка задачи}

Задачу Коши для данного дифференциального уравнения второго порядка преобразовать к задаче Коши для системы двух дифференциальных уравнений первого порядка. Найти решение последней задачи на отрезке $[a, b]$ с шагом $\tau = 0.05$ указанными методами. Оценить погрешность численного решения с помощью правила Рунге (для одношаговых методов). Сравнить численное решение с известным аналитическим решением $u(t)$.

\item%
\underline{Входные данные}

\includegraphics[scale=1.0]{data.png}

\item%
\underline{Краткие теоретические сведения}

Данное ДУ второго порядка преобразуем к задаче Коши для системы двух ДУ первого порядка c помощью следующей замены:

$\begin{cases} 
U_1(t) = U(t) \\ U_2(t) = U'_1(t) = U'(t) 
\end{cases}$

Получим задачу Коши:

$\begin{cases} 
U'_1(t) = f_1(t, U_1(t), U_2(t)) = U_2(t) \\ U'_2(t) = f_2(t, U_1(t), U_2(t)) = -t U_2(t) - \frac{t^2-2}{t^2} U_1(t) \\ U_1(1) = 1,\,  U_2(1)=-1
\end{cases}$

\vskip 0.2in

Тогда метод Рунге-Кутта 3 порядка (3 порядок точности) для данной задачи будет иметь следующий вид:

$\begin{cases} 
y_{1,j+1} =  y_{1,j} + \frac{\tau}{6} (k_1 +4k_2+k_3) \\ y_{2,j+1} =  y_{2,j} + \frac{\tau}{6} (q_1 +4q_2+q_3) \\ k_1 = y_{2,j} \\ q_1 = -t_j y_{2,j} -  \frac{t_j^2-2}{t_j^2}   y_{1,j} \\
k_2 =  y_{2,j} + \frac{1}{2} \tau q_1 \\ q_2 = -(t_j + \frac{1}{2} \tau)(y_{2,j} + \frac{1}{2} \tau q_1) -  \frac{(t_j+ \frac{1}{2} \tau)^2-2}{(t_j+ \frac{1}{2} \tau)^2}  (y_{1,j} + \frac{1}{2} \tau k_1) \\
k_3 =  y_{2,j} - \tau q_1 + 2\tau q_2 \\ q_3 = -(t_j + \tau)( y_{2,j} - \tau q_1 + 2\tau q_2) -  \frac{(t_j+ \tau)^2-2}{(t_j+ \tau)^2}  (y_{1,j} - \tau k_1 + 2\tau k_2) \\ y_{1,0} = 1, \, y_{2,0} = -1, \, j = \overline{0,19}
\end{cases}$

\vskip 0.2in

Экстраполяционный метод Адамса 3 порядка (3 порядок точности):

$\begin{cases} 
y_{1,j+1} =  y_{1,j} + \frac{\tau}{12} (23y_{2,j} - 16y_{2,j-1} + 5y_{2,j-2}) \\ y_{2,j+1} =  y_{2,j} + \frac{\tau}{12} (23(-t_j y_{2,j} -  \frac{t_j^2-2}{t_j^2}   y_{1,j}) - 16(-t_{j-1} y_{2,j-1} -  \frac{t_{j-1}^2-2}{t_{j-1}^2}   y_{1,j-1}) + 5(-t_{j-2} y_{2,j-2} -  \frac{t_{j-2}^2-2}{t_{j-2}^2}   y_{1,j-2}) \\ j = \overline{2,19}
\end{cases}$

Метод является многостадийным, поэтому значения в первых трёх точках берем из метода Рунге-Кутта.

\vskip 0.2in

Неявный метод трапеций (2 порядок точности):

$\begin{cases} 
y_{1,j+1} =  y_{1,j} + \frac{\tau}{2} (y_{2,j} + y_{2,j+1}) \\ y_{2,j+1} =  y_{2,j} + \frac{\tau}{2} (-t_j y_{2,j} -  \frac{t_j^2-2}{t_j^2}   y_{1,j} - t_{j+1} y_{2,j+1} -  \frac{t_{j+1}^2-2}{t_{j+1}^2}   y_{1,j+1}) \\ j = \overline{0,19}
\end{cases}$

В неявном методе трапеций на каждой итерации возникает ситема уравнений относительно $y_{1,j+1}$ и $y_{2,j+1}$, однако, её можно легко разрешить, воспользовавшись методом Гаусса.

\item%
\underline{Листинг программы}

\begin{verbatim}
import numpy as np
import pandas as pd
from matplotlib import pyplot as plt

a, b, step, step2 = (1.0, 2.0, 0.05, 0.1)
u1_0, u2_0 = (1, -1)
n = int((b-a)/step)+1
n2 = int((b-a)/step2)+1

t = np.linspace(a, b, n)
t2 = np.linspace(a, b, n2)
y1_0 = 1/t
y2_0 = -1/(t**2)

y1r = np.zeros(n, dtype = float)
y1a = np.zeros(n, dtype = float)
y1t = np.zeros(n, dtype = float)
y1r_2 = np.zeros(n2, dtype = float)
y1t_2 = np.zeros(n2, dtype = float)
y1r[0], y1a[0], y1t[0], y1r_2[0], y1t_2[0] = u1_0, u1_0, u1_0, u1_0, u1_0

y2r = np.zeros(n, dtype = float)
y2a = np.zeros(n, dtype = float)
y2t = np.zeros(n, dtype = float)
y2r_2 = np.zeros(n2, dtype = float)
y2t_2 = np.zeros(n2, dtype = float)
y2r[0], y2a[0], y2t[0], y2r_2[0], y2t_2[0] = u2_0, u2_0, u2_0, u2_0, u2_0


# ## Part 2: Implementing mathematical methods

def fx(x):
    return (x**2 - 2)/x**2


def f1(t, y1, y2):
    return y2

def f2(t, y1, y2):
    return -t*y2 - ((t**2 - 2)/t**2)*y1


# #### 2.1: Runge–Kutta method

def runge_iteration_y(t, y1, y2, j, step):
    k1 = f1(t[j], y1[j], y2[j])
    q1 = f2(t[j], y1[j], y2[j])
    k2 = f1(t[j] + step/2, y1[j] + k1*step/2, y2[j] + q1*step/2)
    q2 = f2(t[j] + step/2, y1[j] + k1*step/2, y2[j] + q1*step/2)
    k3 = f1(t[j] + step, y1[j] - k1*step + 2*k2*step, y2[j] - q1*step + 2*q2*step)
    q3 = f2(t[j] + step, y1[j] - k1*step + 2*k2*step, y2[j] - q1*step + 2*q2*step)
    return (y1[j] + (step/6)*(k1 + 4*k2 + k3), y2[j] + (step/6)*(q1 + 4*q2 + q3))

# #### 2.2: Adams' method

def adams_iteration_y1(t, y1, y2, j, step):
    return y1[j] + (step/12)*(23 * f1(t[j], y1[j], y2[j]) - 
                              16 * f1(t[j-1], y1[j-1], y2[j-1]) + 
                              5 * f1(t[j-2], y1[j-2], y2[j-2]))



def adams_iteration_y2(t, y1, y2, j, step):
    return y2[j] + (step/12)*(23 * f2(t[j], y1[j], y2[j]) - 
                              16 * f2(t[j-1], y1[j-1], y2[j-1]) + 
                              5 * f2(t[j-2], y1[j-2], y2[j-2]))


							  
# #### 2.3: Trapezium method

def trapezium_iteration(t, y1, y2, j, step):
    a1 = step/2
    b1 = y1[j] + step*y2[j]/2
    a2 = step*t[j+1]/2
    b2 = y2[j] - step*t[j]*y2[j]/2 - step*fx(t[j])*y1[j]/2
    c2 = step*fx(t[j+1])/2
    
    x2 = (b2 - c2*b1)/(1 + a2 + c2*a1)
    x1 = b1 + a1*x2
    return (x1, x2)


# ## Part 3: Value table calculation 

for j in range(1, n):
    y1r[j],y2r[j] = runge_iteration_y1(t, y1r, y2r, j-1, step)
    
for j in range(1, n2):
    y1r_2[j], y2r_2[j] = runge_iteration_y1(t2, y1r_2, y2r_2, j-1, step2)

for j in range(1, 3):
    y1a[j] = y1r[j]
    y2a[j] = y2r[j]

for j in range(3, n):
    y1a[j] = adams_iteration_y1(t, y1a, y2a, j-1, step)
    y2a[j] = adams_iteration_y2(t, y1a, y2a, j-1, step)

for j in range(1, n):
    y1t[j], y2t[j] = trapezium_iteration(t, y1t, y2t, j-1, step)
    
for j in range(1, n2):
    y1t_2[j], y2t_2[j] = trapezium_iteration(t2, y1t_2, y2t_2, j-1, step2)


# ## Part 4: Error calculation

err_r_1 = max(max(np.absolute(y1r[::2] - y1r_2)), max(np.absolute(y2r[::2] - y2r_2))) / 7

err_t_1 = max(max(np.absolute(y1t[::2] - y1t_2)), max(np.absolute(y2t[::2] - y2t_2))) / 3

err_r_2 = max(max(np.absolute(y1_0 - y1r)), max(np.absolute(y2_0 - y2r)))

err_a_2 = max(max(np.absolute(y1_0 - y1a)), max(np.absolute(y2_0 - y2a)))

err_t_2 = max(max(np.absolute(y1_0 - y1t)), max(np.absolute(y2_0 - y2t)))

# ## Part 5: Export results

d = {'t_j':t, 'y1_0':y1_0, 'y2_0':y2_0, 'y1r':y1r, 'y2r':y2r, 'y1a':y1a, 'y2a':y2a, 'y1t':y1t, 'y2t':y2t}

df = pd.DataFrame(data = d)
df = df.to_excel("output2.xlsx")
\end{verbatim}

\item%
\underline{Результаты работы}

\includegraphics[scale=1.0]{results.png}

\vskip 0.2in

\includegraphics[scale=1.0]{err.png}

\vskip 0.2in

\item%
\underline{Выводы}

Как можно увидеть из оценки погрешностей, рассмотренные методы действительно обеспечивают заявленные порядки точности при нахождении приближенного решения задачи Коши для ОДУ.
\end{enumerate}%
\end{document}